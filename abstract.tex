\documentclass[12pt,a4paper,notitlepage,english]{article}

\usepackage[utf8]{inputenc}
\usepackage[T1]{fontenc}
\usepackage{babel}

\usepackage{fullpage}

\begin{document}

\title{Spline based mesh generator for wind turbine blades}
\author{
  E. Fonn
  \thanks{\texttt{eivind.fonn@sintef.no}}
  \thanks{Applied Mathematics, SINTEF ICT, Strindvegen 4, 7034 Trondheim, Norway}
  \and
  A. Rasheed \footnotemark[2]
  \and
  A. M. Kvarving \footnotemark[2]
  \and
  T. Kvamsdal \footnotemark[2]
}

\pagenumbering{gobble}
\maketitle

\abstract{
  Mesh generation involving complex geometries like wind turbines is highly complicated.  The
  problem is generally addressed using tetrahedral meshes or hybrid meshes involving hexahedral
  elements close to the wall and tetrahedral elements elsewhere.  The popularity of such mesh
  generators is attributed to the associated ease of use and automation, though at the cost of
  numerical accuracy in subsequent analysis.  Also, while using such a mesh one never works with the
  actual geometry. Since 2005, Isogeometric Analysis---which offers an integration of analysis and
  CAD geometry through the same basis function, is catching up.  It has been demonstrated that the
  method offers better accuracy and exact geometric representation during analysis.  Since its
  inception, the method has been applied to solve problems from the domain of fluid and structural
  mechanics.  The availability of NURB based surface modelling softwares like Rhinoceros has made it
  possible to create complex geometries with relative ease but a lack of volumetric mesh generator
  based on splines proves to be a bottleneck.

  This work presents an automatic block structured spline based mesh generator for wind turbine
  blades that is being developed to streamline the work flow from CAD modelling to final simulation
  and analysis.  The blade geometry is defined via cross section wing profiles at various points
  along the blade axis, which are then interpolated.  These cross sections have a sharp trailing
  edge, which traditionally forces the use of a ``C-type'' surrounding mesh.  Instead, we modify the
  profiles slightly to produce a rounded trailing edge, allowing a much more efficient ``O-type''
  mesh.  The surrounding space is then meshed using a locally orthogonal generalization of
  transfinite interpolation (TFI).  This is particularly challenging at the tip of the blade, which
  takes the shape of a strongly distorted hemisphere, forcing the implementation of higher
  dimensional TFI.
}


\end{document}
